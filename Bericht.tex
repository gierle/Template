%%%%%%%%%%%%%%%%%%%%%%%%%%%%%%%%%%%%%%%%%%%%%%%%%%%%%%%%%%%%%%%%%%%%%%%%%%%%%%%
%% Inputs
%%%%%%%%%%%%%%%%%%%%%%%%%%%%%%%%%%%%%%%%%%%%%%%%%%%%%%%%%%%%%%%%%%%%%%%%%%%%%%%

\documentclass[
ngerman          			% neue deutsche Rechtschreibung
,a4paper          			% Papiergrösse
%,twoside          			% Zweiseitiger Druck (rechts/links)
%,10pt             			% Schriftgrösse
%,11pt
, 12pt
,pdftex
%,disable         			% Todo-Markierungen auschalten
]{article}
\usepackage[utf8]{inputenc}
\usepackage[ngerman]{babel}
\usepackage[T1]{fontenc}
\usepackage{amsmath}
\usepackage{amsfonts}
%\usepackage{subfigure}
\usepackage{framed}
\usepackage{booktabs}
\usepackage{setspace}
\usepackage{threeparttable}
\usepackage{amssymb}
\usepackage{graphicx}
\usepackage{fancyhdr}
\usepackage[hyphens]{url}
\usepackage[printonlyused,withpage]{acronym}
\usepackage[
	hyperref=true,			% Klickbare Referenzen in der PDF-Datei
	backref=true,			% In der Literaturref. die Seiten angeben, wo ein \cite dazu steht
	bibencoding=inputenc,	% s. inputenc-Paket
	style=ieee,
	backend=bibtex,
	sorting=nty]{biblatex}
%\addbibresource{Bericht.bib}
\let\cite\parencite

\usepackage{icomma}
\usepackage{float}
\usepackage{pdfpages}
\usepackage{hyperref}
\usepackage{csquotes}
\usepackage{listings}
\usepackage[left=2.5cm,right=2.5cm,top=2cm,bottom=3.5cm]{geometry}
\onehalfspacing 			% Zeilenabstand


%%%%%%%%%%%%%%%%%%%%%%%%%%%%%%%%%%%%%%%%%%%%%%%%%%%%%%%%%%%%%%%%%%%%%%%%%%%%%%%
%% Angaben zur Arbeit
%%%%%%%%%%%%%%%%%%%%%%%%%%%%%%%%%%%%%%%%%%%%%%%%%%%%%%%%%%%%%%%%%%%%%%%%%%%%%%%

\newcommand{\Autor}{Max Musternann}
\newcommand{\MatrikelNummer}{XXXXXXX}
\newcommand{\Kursbezeichnung}{TINF1XBX}

\newcommand{\FirmenName}{Musterfirma}
\newcommand{\FirmenStadt}{Musterort}
\newcommand{\FirmenLogoDeckblatt}{\includegraphics[width=2cm]{logo}}

\newcommand{\BetreuerFirma}{Moritz Mustermann}
\newcommand{\BetreuerDHBW}{Dodge}

\newcommand{\Was}{Bachelor-/Studien-/Projektarbeit}

\newcommand{\Titel}{Hier steht ein Mustertitel meiner Bachelor-/Studien-/Projektarbeit}
\newcommand{\AbgabeDatum}{xx. November 109 B.C.}

\newcommand{\Dauer}{von wann bis wann}
\newcommand{\Abschluss}{Bachelor of Science}

\newcommand{\Studiengang}{Informatik / Informationstechnik}

\hypersetup{%%
	pdfauthor={\Autor},
	pdftitle={\Titel},
	pdfsubject={\Was}
}

%%%%%%%%%%%%%%%%%%%%%%%%%%%%%%%%%%%%%%%%%%%%%%%%%%%%%%%%%%%%%%%%%%%%%%%%%%%%%%%

\bibliography{Bericht}

%%%%%%%%%%%%%%%%%%%%%%%%%%%%%%%%%%%%%%%%%%%%%%%%%%%%%%%%%%%%%%%%%%%%%%%%%%%%%%%
%% Dokument
%%%%%%%%%%%%%%%%%%%%%%%%%%%%%%%%%%%%%%%%%%%%%%%%%%%%%%%%%%%%%%%%%%%%%%%%%%%%%%%


\begin{document}
	
	
	\begin{titlepage}
		\begin{center}
			\vspace*{-2cm}
			\FirmenLogoDeckblatt\hfill\includegraphics[width=4cm]{dhbw-logo}\\[2cm]
			{\Huge \Titel}\\[1cm]
			{\Huge\scshape \Was}\\[1cm]
			{\large für die Prüfung zum}\\[0.5cm]
			{\Large \Abschluss}\\[0.5cm]
			{\large des Studienganges \Studiengang}\\[0.5cm]
			{\large an der}\\[0.5cm]
			{\large Dualen Hochschule Baden-Württemberg Karlsruhe}\\[0.5cm]
			{\large von}\\[0.5cm]
			{\large\bfseries \Autor}\\[1cm]
			{\large Abgabedatum \AbgabeDatum}
			\vfill
		\end{center}
		\begin{tabular}{l@{\hspace{2cm}}l}
			Bearbeitungszeitraum			& \Dauer			\\
			Matrikelnummer					& \MatrikelNummer	\\
			Kurs							& \Kursbezeichnung	\\
			Ausbildungsfirma				& \FirmenName		\\
											& \FirmenStadt		\\
			Betreuer der Ausbildungsfirma	& \BetreuerFirma	\\
			Gutachter der Studienakademie	& \BetreuerDHBW		\\
		\end{tabular}
	\end{titlepage}

	%%%%%%%%%%%%%%%%%%%%%%%%%%%%%%%%%%%%%%%%%%%%%%%%%%%%%%%%%%%%%%%%%%%%%%%%%%%%%%%
	%%%%%%%% Erklärung %%%%%%%%
	
	\input{Erklaerung.tex}
	
	%%%%%%%% Verzeichnisse %%%%%%%%
	
	\pagenumbering{Roman}
	
	\clearpage
	\tableofcontents			% Inhaltsverzeichnis hier ausgeben
	\clearpage
	\listoffigures				% Liste der Abbildungen
	\clearpage
	\listoftables
	%\clearpage					% Liste der Tabellen
	%\listofequations			% Liste der Formeln
	\clearpage
	\section*{Abkürzungsverzeichnis}   
\begin{acronym}[DHBW]
    \acro{API}[API]{Application Programming Interface}
    \acro{REST}[REST]{Representational State Transfer}
    \acro{HTTP}[HTTP]{Hypertext Transfer Protocol}
\end{acronym}				% Abkürzungsverzeichnis
	
	%%%%%%%% Abstract %%%%%%%%
	\selectlanguage{ngerman}
	\begin{abstract}
		\input{Abstract/Abstrakt.tex}
		\thispagestyle{empty}
	\end{abstract}
	\clearpage
%	\selectlanguage{english}
%	\begin{abstract}
%		\input{Abstract/Abstract.tex}
%		\thispagestyle{empty}
%	\end{abstract}
%	\clearpage
%	\selectlanguage{ngerman}
	
	%%%%%%%% Styling %%%%%%%%
	
	\pagenumbering{arabic}
	\setcounter{page}{1}
	
	\pagestyle{fancy}
	\fancyhf{}
	\setlength{\headheight}{35pt}
	
	\cfoot{\thepage}
	
	%%%%%%%% Kapitel %%%%%%%%
	
	\clearpage
	\section{Einleitung}
	\label{sec:Einleitung}
	Hier steht meine Einleitung \cite{template.2021}

\begin{itemize}
	\item gls \gls{report}
	\item acrlong \acrlong{ros}
	\item acrshort \acrshort{ros}
	\item acrfull \acrfull{ros}
\end{itemize}
	
	\clearpage
	\section{Grundlagen}
	\label{sec:Grundlagen}
	%\input{Grundlagen/Grundlagen.tex}
	
	\clearpage
	\section{Umsetzung}
	\label{sec:Umsetzung}
	%\input{Umsetzung/Umsetzung.tex}
	
	\clearpage
	\section{Ausblick}
	\label{sec:Ausblick}
	%\input{Ausblick/Ausblick.tex}
	
	%%%%%%%% Anhang %%%%%%%%
	
	\clearpage
	\pagenumbering{Roman}
	\setcounter{page}{8}	% passende Zahl einsetzen
	
	\appendix
	\section{test}
	
	%%%%%%%% Literaturverzeichnis %%%%%%%%
	\clearpage
	\addcontentsline{toc}{section}{Literaturverzeichnis}
	
	\def\refname{Literaturverzeichnis}
	\printbibliography
\end{document}

%%%%%%%%%%%%%%%%%%%% Frequently used commands %%%%%%%%%%%%%%%%%%%%


%\noindent
%\hangindent1cm
%\textbf{Keyword} und der dazugehörige, eingerückte Text


%\begin{figure}[tbt]
%	\begin{center}
%		\includegraphics[width=8cm]{images/}
%	\end{center}
%	\caption[Das steht im Verzeichnis]{Das steht unterm Bild (Abbildung aus \cite{Quelle})}
%	\label{fig:figure}
%\end{figure}